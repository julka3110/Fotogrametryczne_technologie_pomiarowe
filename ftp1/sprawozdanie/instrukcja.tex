\documentclass{article}
\usepackage[polish]{babel}
\usepackage[T1]{fontenc}
\usepackage{float}
\usepackage{subcaption}
\usepackage{caption}
\usepackage[a4paper,top=2cm,bottom=2cm,left=3cm,right=3cm,marginparwidth=1.75cm]{geometry}
\parindent = 0pt
\usepackage{amsmath}
\usepackage{graphicx}
\usepackage{hyperref}

\title{Przetwarzanie danych w Agisoft Metashape}
\author{Julia Gomulska (328936)}

\begin{document}
\maketitle

\section{Interfejs programu}

\begin{figure}[H]
    \centering
    \includegraphics[width=0.85\textwidth]{Interfejs.png}
    \caption{Interfejs programu}
\end{figure}

\section{Działanie programu}

Po uruchomieniu programu należy kolejno:
\begin{enumerate}
    \item Nacisnąć przycisk \textit{Select Photos Directory} i wybrać folder ze zdjęciami
    \item Nacisnąć przycisk \textit{Select Markers File} i wybrać plik z markerami
    \item Nacisnąć przycisk \textit{Select CRS for Markers File} i wybrać układ współrzędnych, w którym zostały zapisane współrzędne markerów; do tego układu zostaną przeliczone zdjęcia.
    \item Wybrać parametry dotyczące wyrównania zdjęć, chmury punktów i modelu.
    \item Nacisnąć przycisk \textit{Do all for 3} i poczekać na zakończenie przetwarzania, którego efektem będzie chmura punktów i model zapisane w katalogu ze zdjęciami.
    \item Nacisnąć przycisk \textit{Start measuring markers} i pomierzyć trzy markery na trzech zdjęciach. Należy zwrócić uwagę na to, by markery nie były zlokalizowane obok siebie, lecz w większej odległości.
    \item Nacisnąć przycisk \textit{Do all for 4  } i poczekać na zakończenie przetwarzania. Nastapi automatyczne wykrycie markerów oraz przypisanie im współrzędnych.
    Współrzędne zostaną porównane ze współrzędnymi z pliku, a odpowiednie markery usunięte. Wynikiem działania jest plik tekstowy zawierający elementy orientacji zewnętrznej,
    zapisany w katalogu ze zdjęciami.
\end{enumerate}

\end{document}